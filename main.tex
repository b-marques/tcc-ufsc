%% abtex2-modelo-trabalho-academico.tex, v-1.9.6 laurocesar
%% Copyright 2012-2016 by abnTeX2 group at http://www.abntex.net.br/ 
%%
%% This work may be distributed and/or modified under the
%% conditions of the LaTeX Project Public License, either version 1.3
%% of this license or (at your option) any later version.
%% The latest version of this license is in
%%   http://www.latex-project.org/lppl.txt
%% and version 1.3 or later is part of all distributions of LaTeX
%% version 2005/12/01 or later.
%%
%% This work has the LPPL maintenance status `maintained'.
%% 
%% The Current Maintainer of this work is the abnTeX2 team, led
%% by Lauro César Araujo. Further information are available on 
%% http://www.abntex.net.br/
%%
%% This work consists of the files abntex2-modelo-trabalho-academico.tex,
%% abntex2-modelo-include-comandos and abntex2-modelo-references.bib
%%

% ------------------------------------------------------------------------
% ------------------------------------------------------------------------
% abnTeX2: Modelo de Trabalho Academico (tese de doutorado, dissertacao de
% mestrado e trabalhos monograficos em geral) em conformidade com 
% ABNT NBR 14724:2011: Informacao e documentacao - Trabalhos academicos -
% Apresentacao
% ------------------------------------------------------------------------
% ------------------------------------------------------------------------

\documentclass[a5paper]{ufsc-thesis}

% \documentclass[
% 	% -- opções da classe memoir --
% 	12pt,				% tamanho da fonte
% 	openright,			% capítulos começam em pág ímpar (insere página vazia caso preciso)
% 	twoside,			% para impressão em recto e verso. Oposto a oneside
% 	a4paper,			% tamanho do papel. 
% 	% -- opções da classe abntex2 --
% 	%chapter=TITLE,		% títulos de capítulos convertidos em letras maiúsculas
% 	%section=TITLE,		% títulos de seções convertidos em letras maiúsculas
% 	%subsection=TITLE,	% títulos de subseções convertidos em letras maiúsculas
% 	%subsubsection=TITLE,% títulos de subsubseções convertidos em letras maiúsculas
% 	% -- opções do pacote babel --
% 	english,			% idioma adicional para hifenização
% 	french,				% idioma adicional para hifenização
% 	spanish,			% idioma adicional para hifenização
% 	brazil				% o último idioma é o principal do documento
% 	]{abntex2}


% ---
% Pacotes básicos 
% ---
\usepackage{lmodern}			% Usa a fonte Latin Modern			
\usepackage[T1]{fontenc}		% Selecao de codigos de fonte.
\usepackage[utf8]{inputenc}		% Codificacao do documento (conversão automática dos acentos)
\usepackage{lastpage}			% Usado pela Ficha catalográfica
\usepackage{indentfirst}		% Indenta o primeiro parágrafo de cada seção.
\usepackage{color}				% Controle das cores
\usepackage{graphicx}			% Inclusão de gráficos
\usepackage{microtype} 			% para melhorias de justificação
% ---
		
% ---
% Pacotes adicionais, usados apenas no âmbito do Modelo Canônico do abnteX2
% ---
\usepackage{lipsum}				% para geração de dummy text
\usepackage{todonotes}          % todos
\usepackage{pdfpages}
% \usepackage{subfigure}

% ---

% ---
% Pacotes de citações
% ---
\usepackage[brazilian,hyperpageref]{backref}	 % Paginas com as citações na bibl
\usepackage[alf]{abntex2cite}	% Citações padrão ABNT

% --- 
% CONFIGURAÇÕES DE PACOTES
% --- 
\usepackage{xspace}

\newcommand{\hpc}{HPC\xspace}
\newcommand{\io}{E/S\xspace}
\newcommand{\noc}{NoC\xspace}
\newcommand{\nocs}{NoCs\xspace}
\newcommand{\cpu}{CPU\xspace}
\newcommand{\gpu}{GPU\xspace}
\newcommand{\api}{API\xspace}
\newcommand{\ipc}{IPC\xspace}
\newcommand{\lpddr}{LPDDR3\xspace}
\newcommand{\ram}{RAM\xspace}
\newcommand{\uma}{UMA\xspace}
\newcommand{\numa}{NUMA\xspace}

\newcommand{\mppa}{MPPA-256\xspace}
\newcommand{\fw}{\textit{framework}\xspace}
\newcommand{\fws}{\textit{frameworks}\xspace}
\newcommand{\manycore}{\textit{manycore}\xspace}
\newcommand{\Manycore}{\textit{Manycore}\xspace}
\newcommand{\multicore}{\textit{multicore}\xspace}
\newcommand{\pskel}{PSkel\xspace}
\newcommand{\pe}{\textit{Processing Element}\xspace}
\newcommand{\pes}{\textit{Processing Elements}\xspace}
\newcommand{\rman}{\textit{Resource Manager}\xspace}
\newcommand{\pskelmppa}{PSkel-MPPA\xspace}
\newcommand{\cpus}{CPUs\xspace}
\newcommand{\chip}{\textit{chip}\xspace}
\newcommand{\chips}{\textit{chips}\xspace}
\newcommand{\gpus}{GPUs\xspace}
\newcommand{\etal}{\textit{et al.}\xspace}
\newcommand{\async}{ASYNC\xspace}
\newcommand{\cluster}{\textit{cluster}\xspace}
\newcommand{\clusters}{\textit{clusters}\xspace}
\newcommand{\thread}{\textit{thread}\xspace}
\newcommand{\threads}{\textit{threads}\xspace}
\newcommand{\cache}{\textit{cache}\xspace}
\newcommand{\caches}{\textit{caches}\xspace}
\newcommand{\tile}{\textit{tile}\xspace}
\newcommand{\tiles}{\textit{tiles}\xspace}
\newcommand{\kernel}{\textit{kernel}\xspace}

\newcommand{\xeon}{\textit{Xeon E5}\xspace}
\newcommand{\rapl}{\textit{RAPL}\xspace}
\newcommand{\tesla}{\textit{Tesla K40}\xspace}
\newcommand{\nvml}{\textit{NVML}\xspace}
\newcommand{\eg}{\textit{e.g.,}\xspace}
\newcommand{\ie}{\textit{i.e.,}\xspace}

\newcommand{\convolution}{\textsc{Convolution}\xspace}
\newcommand{\gol}{\textsc{GoL}\xspace}
\newcommand{\jacobi}{\textsc{Jacobi}\xspace}

\newcommand{\ina}{$2048$x$2048$\xspace}
\newcommand{\inb}{$4096$x$4096$\xspace}
\newcommand{\inc}{$8192$x$8192$\xspace}
\newcommand{\ind}{$12288$x$12288$\xspace}

\newcommand{\tilea}{$32$x$32$\xspace}
\newcommand{\tileb}{$64$x$64$\xspace}
\newcommand{\tilec}{$128$x$128$\xspace}
\newcommand{\tiled}{$256$x$256$\xspace}

\newcommand{\hw}{\textit{hardware}\xspace}


% ---
% Configurações do pacote backref
% Usado sem a opção hyperpageref de backref
\renewcommand{\backrefpagesname}{Citado na(s) página(s):~}
% Texto padrão antes do número das páginas
\renewcommand{\backref}{}
% Define os textos da citação
\renewcommand*{\backrefalt}[4]{
	\ifcase #1 %
		Nenhuma citação no texto.%
	\or
		Citado na página #2.%
	\else
		Citado #1 vezes nas páginas #2.%
	\fi}%
% ---

% ---
% Informações de dados para CAPA e FOLHA DE ROSTO
% ---
\titulo{OTIMIZAÇÃO DO \textit{FRAMEWORK} PSKEL PARA O PROCESSADOR \textit{MANYCORE} MPPA-256}
\autor{Bruno Marques do Nascimento}
\local{Florianópolis}
\data{2018}
\orientador{Prof. Dr. Márcio Bastos Castro}
\instituicao{%
  UNIVERSIDADE FEDERAL DE SANTA CATARINA -- UFSC
  \par
  DEPARTAMENTO DE INFORMÁTICA E ESTATÍSTICA}
\tipotrabalho{Trabalho de conclusão de curso de graduação}
% O preambulo deve conter o tipo do trabalho, o objetivo, 
% o nome da instituição e a área de concentração 
\preambulo{Trabalho de Conclusão de Curso submetido ao Curso de Bacharelado em
Ciência da Computação para a obtenção do Grau de Bacharel em Ciência da Computação.}
% ---


% ---
% Configurações de aparência do PDF final

% alterando o aspecto da cor azul
\definecolor{blue}{RGB}{41,5,195}

% informações do PDF
\makeatletter

% % set section title to capital letters in summary
% \let\oldcontentsline\contentsline
% \def\contentsline#1#2{%
%   \expandafter\ifx\csname l@#1\endcsname\l@section
%     \expandafter\@firstoftwo
%   \else
%     \expandafter\@secondoftwo
%   \fi
%   {%
%     \oldcontentsline{#1}{\MakeTextUppercase{#2}}%
%   }{%
%     \oldcontentsline{#1}{#2}%
%   }%
% }

\hypersetup{
     	%pagebackref=true,
		pdftitle={\@title}, 
		pdfauthor={\@author},
    	pdfsubject={\imprimirpreambulo},
	    pdfcreator={LaTeX with abnTeX2},
		pdfkeywords={abnt}{latex}{abntex}{abntex2}{trabalho acadêmico}, 
		colorlinks=true,       		% false: boxed links; true: colored links
    	linkcolor=black,          	% color of internal links
    	citecolor=black,        		% color of links to bibliography
    	filecolor=black,      		% color of file links
		urlcolor=black,
		bookmarksdepth=4
}
\makeatother
% --- 

% --- 
% Espaçamentos entre linhas e parágrafos 
% --- 

% O tamanho do parágrafo é dado por:
\setlength{\parindent}{1.3cm}

% Controle do espaçamento entre um parágrafo e outro:
\setlength{\parskip}{0.2cm}  % tente também \onelineskip

% ---
% compila o indice
% ---
\makeindex
% ---

% ----
% Início do documento
% ----
\begin{document}


% Seleciona o idioma do documento (conforme pacotes do babel)
%\selectlanguage{english}
\selectlanguage{brazil}

% Retira espaço extra obsoleto entre as frases.
\frenchspacing 

% ----------------------------------------------------------
% ELEMENTOS PRÉ-TEXTUAIS
% ----------------------------------------------------------
% \pretextual

% ---
% Capa
% ---
\imprimircapa
% ---

% ---
% Folha de rosto
% (o * indica que haverá a ficha bibliográfica)
% ---
\imprimirfolhaderosto*
% ---

% ---
% Inserir a ficha bibliografica
% ---

% Isto é um exemplo de Ficha Catalográfica, ou ``Dados internacionais de
% catalogação-na-publicação''. Você pode utilizar este modelo como referência. 
% Porém, provavelmente a biblioteca da sua universidade lhe fornecerá um PDF
% com a ficha catalográfica definitiva após a defesa do trabalho. Quando estiver
% com o documento, salve-o como PDF no diretório do seu projeto e substitua todo
% o conteúdo de implementação deste arquivo pelo comando abaixo:
%
\begin{fichacatalografica}
    \includepdf{Ficha_Catalografica.pdf}
\end{fichacatalografica}

% \imprimirfichacatalografica%

% ---


% ---

% ---
% Inserir folha de aprovação
% ---

% Isto é um exemplo de Folha de aprovação, elemento obrigatório da NBR
% 14724/2011 (seção 4.2.1.3). Você pode utilizar este modelo até a aprovação
% do trabalho. Após isso, substitua todo o conteúdo deste arquivo por uma
% imagem da página assinada pela banca com o comando abaixo:
%
% \includepdf{folhadeaprovacao_final.pdf}
%
\begin{folhadeaprovacao}

	\begin{center}
		{\ABNTEXchapterfont\large\imprimirautor}
		
		\vspace*{\fill}
		\begin{center}
			\ABNTEXchapterfont\bfseries\Large\imprimirtitulo
		\end{center}
		\vspace*{\fill}

		\hspace{.45\textwidth}
		% \begin{minipage}{.5\textwidth}
		%     \imprimirpreambulo
		% \end{minipage}%
		\vspace*{\fill}
	\end{center}

	Este Trabalho de Conclusão de Curso foi julgado adequado para obtenção do 
	Título de ``Bacharel em Ciência da Computação'' e aprovado em sua forma final
	pelo Curso de Bacharelado em Ciência da Computação.
   
	\begin{center}   
	\imprimirlocal, 27 de novembro de 2018.
	\end{center}
	\assinatura{\textbf{Prof. Dr. Renato Cislagh} \\ Coordenador} 


	\noindent\textbf{\\Banca examinadora:}

	\assinatura{\textbf{\imprimirorientador} \\ 
		Orientador \\ Universidade Federal de Santa Catarina} 
	\assinatura{\textbf{Prof. Dr. Frank Augusto Siqueira} \\ Universidade Federal de Santa Catarina}
	\assinatura{\textbf{Prof.ª Dr.ª Patrícia Della Méa Plentz} \\ Universidade Federal de Santa Catarina}
  
\end{folhadeaprovacao}
% ---

% ---
% Dedicatória
% ---
\begin{dedicatoria}
   \vspace*{\fill}
   \centering
   \noindent
   \todo[inline]{\textit{ Este trabalho é dedicado à...}} \vspace*{\fill}
\end{dedicatoria}
% ---

% ---
% Agradecimentos
% ---
\begin{agradecimentos}
	\todo[inline]{ToDo}
\end{agradecimentos}
% ---

% ---
% Epígrafe
% ---
\begin{epigrafe}
    \vspace*{\fill}
	\begin{flushright}
		\todo[inline]{\textit{ToDo}}
		\textit{``Some citation''\\
		(Surname, Name)}
	\end{flushright}
\end{epigrafe}
% ---

% ---
% RESUMOS
% ---


% resumo em português
\setlength{\absparsep}{18pt} % ajusta o espaçamento dos parágrafos do resumo
\begin{resumo}
	Uma nova classe de \emph{chips} altamente paralelos de baixo consumo energéitco que lidam com a restrição de energia foram descobertos. Os processadores Sunway SW26010 e Kalray MPPA-256 são exemplos deles, entregando mais de duzentos núcleos de processamento em um único \emph{chip} de baixo consumo energético. Apesar de apresentarem melhor eficiência energética do que os processadores \emph{multicore} de propósito geral, características arquiteturais como a limitada quantidade de memória ditribuída no \emph{chip} torna o desenvolvimento de aplicações científicas paralelas eficientes uma tarefa desafiadora. Neste projeto foram propostas otimizações ao \fw PSkelMPPA, que prove uma abstração única e de alto nível para programação estêncil no processador \mppa, livrando os programadores de serem responsáveis pela tarefa de explicitamente lidar com a comunicação e com o modelo de programação paralela híbrida do \mppa.

	\textbf{Palavras-chave}: \Manycore. \mppa. Processamento de alto desempenho. Eficiência energética.
\end{resumo}

% resumo em inglês
\begin{resumo}[Abstract]
	\begin{otherlanguage*}{english}
		\todo[inline]{ToDo}

   		\vspace{\onelineskip}
 
   \noindent 
   \textbf{Keywords}: ToDo.
 \end{otherlanguage*}
\end{resumo}



% ---
% inserir lista de ilustrações
% ---
\pdfbookmark[0]{\listfigurename}{lof}
\listoffigures*
\cleardoublepage
% ---

% ---
% inserir lista de tabelas
% ---
% \pdfbookmark[0]{\listtablename}{lot}
% \listoftables*
% \cleardoublepage
% ---

% ---
% inserir lista de abreviaturas e siglas
% ---
\begin{siglas}
  \item[\mppa] \textit{Massively Parallel Processor Array}
  \item[\io] Entrada e Saída
  \item[\noc] \textit{Network-on-Chip}
  \item[\cpu] \textit{Central processing unit}
  \item[\gpu] \textit{Graphics processing unit}
  \item[\api] \textit{Application Programming Interface}
  \item[\ipc] \textit{Inter-Process Communication}
\end{siglas}
% ---

% ---
% inserir lista de símbolos
% ---
% \begin{simbolos}
%   \item[$ \Gamma $] Letra grega Gama
%   \item[$ \Lambda $] Lambda
%   \item[$ \zeta $] Letra grega minúscula zeta
%   \item[$ \in $] Pertence
% \end{simbolos}
% ---

% ---
% inserir o sumario
% ---
\pdfbookmark[0]{\contentsname}{toc}
\tableofcontents*
\cleardoublepage
% ---



% ----------------------------------------------------------
% ELEMENTOS TEXTUAIS
% ----------------------------------------------------------
\textual

% ----------------------------------------------------------
% Introdução
% ----------------------------------------------------------
\chapter[Introdução]{Introdução}

Até a última década, o desempenho das arquiteturas utilizadas na área de Computação de Alto Desempenho (\textit{High Performance Computing} - HPC) tem sido quase exclusivamente quantificado pelo seu poder de processamento. No entanto, a eficiência energética está sendo considerada recentemente tão importante quanto o desempenho e tornou-se um aspecto crítico para o desenvolvimento de sistemas escaláveis~\cite{francesquini:hal-01092325}. Portanto a pesquisa e o desenvolvimento científico nesta área com enfoque na redução do consumo de energia tem se tornado de extrema importância para o avanço tecnológico.

Assim, é notório o surgimento de uma nova classe de processadores, os denominados processadores \textit{manycore}, com mais de centenas de núcleos de processamento e de baixo consumo de energia, capazes de lidar com o paralelismo dados e tarefas, como o \mppa~\cite{castro2013}. 
Apesar dos benefícios oriundos dos processadores \textit{manycore}, eles também trazem desafios para a programação de aplicações paralelas devido às suas características arquiteturais~\cite{castro:hal-01273153}. Uma das formas de simplificar o desenvolvimento de aplicações paralelas, abstraindo os detalhes arquiteturais e de programação de baixo nível, é através da utilização de padrões paralelos ou esqueletos algorítimicos~\cite{COLE2004389}.

Um destes padrões é denominado estêncil. Este padrão consiste em varrer os elementos de um dado \textit{array} de entrada de \textit{n}-dimensões, e modificar o valor de cada elemento com base nos valores dos elementos vizinhos, produzindo assim um \textit{array} de saída de \textit{n}-dimensões com valores modificados. Além disso, essa etapa de varredura e modificação de valores pode ser realizada iterativamente, na qual o \textit{array} de saída de uma iteração, será o \textit{array} de entrada da iteração seguinte. É válido destacar que muitos destes padrões estão presentes nas mais diversas áreas do conhecimento, como física, matemática, processamento de imagens, entre outros~\cite{Holewinski:2012:HCG:2304576.2304619}.

Dentre os \fws propostos, o \pskel se destaca por prover uma abstração para ambientes heterogêneos de programação, que contam com a presença de processadores \textit{multicore} e placas gráficas. Além disso, foram observados ganhos na performace média de até 76\%~\cite{CPE:CPE3479}.

Com a necessidade e importância de cunho científico do desenvolvimento de aplicações paralelas nos processadores \textit{manycore}, foi proposta uma adaptação do \textit{framework} \pskel para que ele dê suporte ao processador \mppa. Os resultados obtidos com o \pskel no \mppa foram promissores, apresentando uma redução no consumo energético de aplicações estêncil quando comparado  com execuções em um processador de alto desempenho Intel Broadwell~\cite{wscad2017}.

Todavia, foi observado que o tempo desperdiçado na comunicação das aplicações do \mppa é elevado e impacta diretamente nos testes e experimentos realizados, comunicação está que ocorre através de \textit{Networks-on-Chip} (NoCs) de dados e de controle. Decorrente desta estrutura, o delay de comunicação entre elementos fisicamente mais distantes na rede será maior do que elementos mais próximos.

Além disso, a atual API de comunicação utilizada no \pskel para o \mppa é similar ao modelo clássico POSIX de baixo nível para \textit{Inter-Process Communication} (IPC), o que dificulta o desenvolvimento de rotinas de comunicação e expõe essas rotinas a possíveis perdas de desempenho não trivialmente detectáveis. A nova API de comunicação assíncrona desenvolvida e disponibilizada pelos desenvolvedores do processador, eleva o nível de abstração dessas rotinas de comunicação com implementações otimizadas para o processador, acarretando em maior robustez e desempenho na utilização da NoC. Com isso, nota-se a importância da revisão e estudo da implementação atual da adaptação do \textit{framework} \pskel para o \mppa, visando encontrar espaços para otimização e ganho de desempenho, que impactarão diretamente no gasto energético do processador, que já se mostrou promissor quando comparado ao Intel Broadwell. 
% ---------------------------------

% ---------------------------------
\section{Objetivos}
\label{sec:objetivos}
A medida que se reduz o número de comunicações e sincronizações, através do aumento do número de elementos processados a cada iteração de uma aplicação de computação estêncil no \mppa é observada a queda no tempo de sua execução e por consequência a redução do consumo de energia~\cite{wscad2017}.

\subsection{Objetivos Gerais}
\label{subsec:objetivos-gerais}

 Portanto, uma vez identificado que os gastos de comunicação são fatores relevantes para a influência do consumo de energia e tempo de execução das aplicações, este \textit{Trabalho de Conclusão de Curso} tem como objetivo geral propor otimizações na comunicação de dados para o \fw \pskel adaptado para o \mppa visando o ganho de desempenho e redução de consumo de energia pelas aplicações paralelas de padrão estêncil no \mppa.

\subsection{Objetivos Especificos}
\label{subsec:objetivos-específicos}

Os objetivos específicos deste trabalho são: 

\begin{enumerate}
\item Estudar a nova API de comunicação assíncrona disponibilizada pelo fabricante do processador e realizar as modificações de código necessárias no \pskel para fazer uso da mesma;
\item Propor e implementar otimizações na comunicação e na distribuição de dados entre os elementos de processamento a fim de se obter ganhos de desempenho e redução no consumo de energia;
\item Realizar experimentos com aplicações estêncil para medir o desempenho e a eficiência energética das otimizações propostas, comparando resultados obtidos com outros processadores \textit{multicore} e processadores gráficos.
\end{enumerate}

% ----------------------------------------------------------
% Fundamentação Teórica
% ----------------------------------------------------------
\chapter[Fundamentação Teórica]{Fundamentação Teórica}

Esta seção apresenta a fundamentação teórica sobre o processador \textit{manycore} \mppa e o \textit{framework} PSkel e sua adaptação utilizada nesse trabalho.

\section{MPPA-256}
\label{subsec:mppa}

O \mppa é um processador \textit{manycore} desenvolvido pela empresa francesa
Kalray. Esse processador possui 256 núcleos e usuário e 32 núcleos de sistema para processamento a 400 MHz. Esses núcleos estão distribuídos entre 16 \textit{clusters} de computação e 4 \textit{clusters} de \textit{Input/Output}(\io), que se comunicam através de NoCs de dados e controle. O processador utilizado no desenvolver deste projeto de pesquisa possui uma memória global de baixa potência (LPDDR3) de 2GB conectada a um dos susistemas de \io. A arquitetura do \mppa é ilustrada na Figura~\ref{fig:mppaOverall}. Cada cluster de computação tem os seguintes componentes:

\begin{itemize}
    \item 16 núcleos chamados de \textit{Processing Elements} (\pes), que são responsáveis por executar as \textit{threads} de usuário (uma \textit{thread} por \pe), e não pode ser interrompida ou preemptada;
    
    \item um \textit{Resource Manager} (\rman), responsável por executar o sistema operacional e gerenciar a comunicação;
    
    \item uma memória compartilhada de baixa latência de 2MB, que permite um grande banda e fluxo e dados e controle entre os \pes presentes no mesmo \textit{cluster} de computação; e
    
    \item dois controladores de NoC, um para dados e outro para controle.
    
\end{itemize}


% \begin{figure}[t]
% 	\centering
% 	\subfigure[\mppa.]{\includegraphics[width=0.3\textwidth]{figs/mppa-overview.pdf}\label{fig:mppaOverall}}
% 	\qquad
% 	\subfigure[O padrão estêncil.]{\includegraphics[width=0.38\textwidth]{figs/stencilComputation.pdf}\label{fig:stencil}}
%     \caption{Visão geral do \mppa e do padrão estêncil~\cite{Castro-Podesta-ERAD:2017}.}
%     \label{fig:config}
% \end{figure}

Trabalhos anteriores mostraram que desenvolver aplicações paralelas otimizadas
para o \mppa é um grande desafio~\cite{Castro-IA3-JPDC:2014} devido a alguns
fatores importantes, tais como: o modelo de memória distribuída presente no
\mppa, a capacidade de memória dentro do \textit{chip} e a comunicação explícita
através da \noc. Mais detalhes sobre esses desafios são apresentados em
~\cite{Castro-Podesta-ERAD:2017}.

\section{Padrão estêncil e PSkel}
\label{subsec:pskel}

O padrão estêncil, ilustrado pela Figura~\ref{fig:stencil}, funciona da
seguinte forma. Para cada elemento de
uma estrutura $n$-dimensional é computado um novo valor relativo aos vizinhos do
elemento atual. Os vizinhos de um elemento são determinados pela máscara da
computação. Por fim, cada novo valor computado é atribuído à sua célula
respectiva em uma estrutura $n$-dimensional de saída. Em aplicações
estêncil iterativas, a estrutura de saída é utilizada como
estrutura de entrada de uma nova iteração da aplicação.

O \pskel é um \fw de programação em alto nível para aplicações baseadas no
padrão estêncil, baseado no conceito de esqueletos paralelos, oferecendo suporte para a execução dessas aplicações em
ambientes heterogêneos, incluindo \cpu e \gpu. \pskel oferece um interface única de programação, desacoplada do \textit{back-end} de execução, permitindo que o usuário se preocupe apenas em implementar o \textit{kernel} estêncil que descreve a computação, enquanto o \fw fica responsável pela tradução das abstrações descritas para código paralelo de baixo nível em C++, gestão de memória e transferência de dados, tudo isso de forma transparente paar o usuário~\cite{pereira15}.


\section{PSkel-MPPA}
\label{subsec:pskel-mppa}

A adaptação PSkel-MPPA, é uma adaptação do \pskel proposta por~\cite{PodestaJr.2017}, ela faz uso de uma \api similar à POSIX \ipc para comunicação, e será tratada como \ipc no decorrer deste relatório. Nela, são utilizados portais de comunicação para o envio de dados e o método de \textit{strides}  para gerenciar explicitamente o envio e recebimento de \textit{tiles}. Essa adaptação possíbilita o uso do \fw PSkel com o processador \textit{manycore} \mppa.



% ----------------------------------------------------------
% Trabalhos relacionados
% ----------------------------------------------------------
% The \phantomsection command is needed to create a link to a place in the document that is not a
% figure, equation, table, section, subsection, chapter, etc.
%
% When do I need to invoke \phantomsection?
% https://tex.stackexchange.com/questions/44088/when-do-i-need-to-invoke-phantomsection
\phantomsection
\chapter{Trabalhos Relacionados}
\phantomsection

Devido a importância dos esqueletos paralelos, e especifcamente o padrão paralelo estêncil, muitos esforços de pesquisas recentes buscam melhorar o desempenho e o suporte desses esqueletos em processadores \manycore. \textit{Buono} \etal\citeyear{buono13} portaram um \fw baseado em esqueletos paralelos, chamado \textit{FastFlow}, para o processador \textit{manycore} TilePro64, que possui 64 núcleos de processamento idênticos, interconectados por uma malha de \noc. Similarmente, \textit{Thorarensen} \etal\citeyear{thoraransen16} apresentaram um novo \textit{back-end} do \fw SkePU para o processador \textit{manycore} Myriad2, que possui como característica uma arquitetura heterogênea, visando dispositivos com restrição de energia e principalmente aplicações de visão computacional. \textit{Gysi} \etal\citeyear{gysi15} propuseram um \fw para otimização automática da repartição de computações estêncil em sistemas híbridos de \cpu e \gpu.

Recentes trabalhos estudaram o desempenho e/ou a eficiência energética de processadores \manycore de baixa potência. \textit{Totoni} \etal\citeyear{SCCEnergy:2012} compararam a potência e o desempenho do \textit{Intel's Single-Chip Cloud Computer} (SCC) com outros tipos de \textit{CPUs} e \textit{GPUs}. Porém, eles mostraram que não existe um solução única que entrega o melhor troca entre potência e performance, os resultados mostram que \textit{manycores} são uma oportunidade para o futuro. \textit{Souza} \etal\citeyear{Castro-Souza-CCPE:2016} propuseram um conjunto de \textit{benchmarks} para avaliar o \mppa \manycore processor. O \textit{benchmark} oferece diversas aplicações que utilizam padrões paralelos, tipos de trabalho, intensidade de comunicação e estratégias de carga de trabalho, adequado para uma ampla compreensão do desempenho e consumo de energia do \mppa e novos \textit{manycores} que estão por vir. \textit{Francesquini} \etal\citeyear{francesquini:hal-01092325} avaliaram três diferentes classes de aplicação (consumo de CPU, consumo de memória e uma composição híbrida dos dois tipos anteriores) utilizando plataformas de alto paralelismo como o \mppa em uma plataforma NUMA de 24 nós e 192 núcleos. Eles mostraram que as arquiteturas \textit{manycore} podem ser competitivas, mesmo se a aplicação é irregular por natureza.

De acordo com relevante conhecimento na área, o \pskelmppa é a primeira implementação completa de um \fw com uso de padrões paralelos no \mppa. A solução proposta \cite{Podesta:TCC} livra os programadores da necessidade de lidar explicitamente com a gestão de comunicação e envio de dados pela \noc, assim como a preocupação de lidar com um ambiente híbrido de execução e a ausência de coerência de \textit{cache} no \mppa.

% ----------------------------------------------------------
% Otimização PSkel-MPPA
% ----------------------------------------------------------
% \chapter[Otimização PSkel-MPPA]{Otimização PSkel-MPPA}


% ----------------------------------------------------------
% Experimento
% ----------------------------------------------------------
% % The \phantomsection command is needed to create a link to a place in the document that is not a
% figure, equation, table, section, subsection, chapter, etc.
%
% When do I need to invoke \phantomsection?
% https://tex.stackexchange.com/questions/44088/when-do-i-need-to-invoke-phantomsection
\phantomsection

% ---
\chapter{Experimentos}
\label{cap:experimentos}
\phantomsection


% Esta seção apresenta os resultados obtidos com a solução proposta, comparando-a com a versão apresentada em~\cite{wscad2017}.
% Todas as métricas foram obtidas com auxílio de ferramentas disponíveis no \mppa.
% Os dados se referem a execução de uma única iteração das aplicações Fur, GoL e Jacobi.
% A aplicação \textbf{\textit{Fur}} realiza a simulação de padrões de pigmento sobre
% pelos de animais. A aplicação \textbf{\textit{GoL}} é um autômato celular que
% implementa o Jogo da Vida de Conway. Por fim, a aplicação
% \textbf{\textit{Jacobi}} implementa o método de Jacobi para a resolução de equações matriciais.
% Foram realizadas $5$ repetições de cada experimento, computando-se a média aritmética dos valores.
% A variabilidade dos valores obtidos foi extremamente pequena (desvio-padrão inferior à $1\%$), pois as \textit{threads} da aplicação são executadas de maneira ininterrupta no \mppa.

% Na análise da escalabilidade da versão ASYNC, apresentada pela
% Figura~\ref{fig:escalabilidade}, foi utilizada uma matriz de tamanho
% $4.096$x$4.096$ e \textit{tiles} de tamanho $128$x$128$. Foi obtido um ganho de
% desempenho de até $15,6$x com $16$ \textit{clusters} em relação à execução com
% um único \textit{cluster} (aplicação Fur). Por outro lado, a
% Figura~\ref{fig:tiles} apresenta o impacto do tamanho dos \textit{tiles} sobre o
% desempenho nessa aplicação. Para obter uma maior precisão nos resultados em
% relação ao tempo de execução foram utilizadas matrizes de tamanho maior que
% $2.048$x$2.048$. Além disso, devido à memória limitada, não foram utilizadas
% matrizes de entrada e \textit{tiles} maiores que $12.288$x$12.288$ e $128$x$128$,
% respectivamente. Os resultados mostram que o tempo de execução da aplicação diminui à medida em que se aumenta o tamanho dos \textit{tiles}, obtendo-se ganhos de até $2$x. Este comportamento é decorrente do melhor aproveitamento da vazão da \noc, realizando menos transferências com maior quantidade de dados. As aplicações GoL e Jacobi também apresentaram comportamentos similares.
% %
% As Figuras~\ref{fig:compara-tempo} e~\ref{fig:compara-energia} apresentam a comparação de tempo e consumo de energia entre as versões ASYNC e IPC. Nesse experimento, foi utilizada uma matriz de entrada de tamanho $12.288$x$12.288$, \textit{tiles} de tamanho $128$x$128$ e $16$ \textit{clusters}. Como pode ser observado, os resultados mostram que o tempo de execução da versão ASYNC é, aproximadamente, $8$x menor em relação ao tempo da versão IPC. O resultado da energia consumida segue um comportamento similar, apresentando uma eficiência energética superior em até $6$x a favor da versão ASYNC.
% A versão IPC possui funções que manipulam dados (no mínimo, $7,1$\% do tempo total) e distribuições de tarefas (no mínimo, $72$\% do tempo total), trazendo um impacto negativo sobre o tempo de execução.

% \begin{figure}[htb]
% 	\centering
% 	\caption{Escalabilidade.}
% 	\includegraphics[width=0.4\textwidth]{figs/MPPAPlotScalabilityAPI.pdf}
% 	\legend{Fonte - o autor}
% 	\label{fig:escalabilidade}
% \end{figure}

% \begin{figure}[htb]
% 	\centering
% 	\caption{\textit{Tiles} vs. tempo (Fur).}
% 	\includegraphics[width=0.4\textwidth]{figs/MPPAPlotAPIfurTimeTiles.pdf}
% 	\legend{Fonte - o autor}
% 	\label{fig:tiles}
% \end{figure}

% \begin{figure}[htb]
% 	\centering
% 	\caption{Tempo.}
% 	\includegraphics[width=0.4\textwidth]{figs/ComparisonTimeTiles1.pdf}
% 	\legend{Fonte - o autor}
% 	\label{fig:compara-tempo}
% \end{figure}

% \begin{figure}[htb]
% 	\centering
% 	\caption{Energia.}
% 	\includegraphics[width=0.4\textwidth]{figs/ComparisonEnergyTiles1.pdf}
% 	\legend{Fonte - o autor}
% 	\label{fig:compara-energia}
% \end{figure}


% ----------------------------------------------------------
% Finaliza a parte no bookmark do PDF
% para que se inicie o bookmark na raiz
% e adiciona espaço de parte no Sumário
% ----------------------------------------------------------
\phantompart

% ----------------------------------------------------------
% Conclusão
% ----------------------------------------------------------
% % The \phantomsection command is needed to create a link to a place in the document that is not a
% figure, equation, table, section, subsection, chapter, etc.
%
% When do I need to invoke \phantomsection?
% https://tex.stackexchange.com/questions/44088/when-do-i-need-to-invoke-phantomsection
\phantomsection

% ---
\chapter{Conclusão}
\label{cap:conclusao}
\phantomsection

É irrefutável a busca pelo ganho de desempenho atrelado ao aumento da eficiência energética. O futuro persegue incessantemente dispositivos de computação cada vez mais potentes e com menor consumo de energia. Neste trabalho essa busca se traduz na otimização de um \textit{framework} desenvolvido para o processador \mppa que é um \chip de computação de alto desempenho voltado ao baixo consumo energético. Através do estudo e exploração de \textit{software} já existente, almejando otimizá-los em ambos os eixos. As propostas e implementações realizadas visam a exploração de possíveis brechas que nos permitam obter resultados significativos.

A nova \api de comunicação assíncrona facilita o desenvolvimento do \textit{back-end} do \fw através do seu alto nível de abstração apresentado, contribuindo também para manutenabilidade do código fonte. Para comprovar a significância do trabalho é necessária a realização de experimentos e comparação com os resultados atuais de desempenho e consumo energético do \pskelmppa. Além disso, continuar buscando possíveis otimizações através do estudo e exploração teórico e prático envolvidos no âmbito deste trabalho.



% O desenvolvimento de aplicações otimizadas para processadores \textit{manycore} de baixa potência é bastante desafiador devido a fatores importantes tais como a existência de um modelo de programação híbrido, capacidade limitada de memória no \textit{chip}, ausência de coerência de \textit{cache}, entre outros. Neste artigo foi apresentada uma nova versão otimizada do \fw \pskel para o processador \mppa. Os resultados mostraram que a nova versão obteve ganhos de desempenho de até $8$x e uma redução no consumo de energia de até $6$x, em comparação com a solução inicial proposta em~\cite{wscad2017}. Como trabalhos futuros, pretende-se comparar o desempenho e consumo de energia com outros processadores e implementar suporte a matrizes tridimensionais.


% ----------------------------------------------------------
% ELEMENTOS PÓS-TEXTUAIS
% ----------------------------------------------------------
\postextual
% ----------------------------------------------------------

% ----------------------------------------------------------
% Referências bibliográficas
% ----------------------------------------------------------
\bibliography{bibliografia}

% ----------------------------------------------------------
% Glossário
% ----------------------------------------------------------
%
% Consulte o manual da classe abntex2 para orientações sobre o glossário.
%
%\glossary

% ----------------------------------------------------------
% Apêndices
% ----------------------------------------------------------

% ---
% Inicia os apêndices
% ---
\begin{apendicesenv}

% Imprime uma página indicando o início dos apêndices
\partapendices

% ----------------------------------------------------------
% Apêndice A
% ----------------------------------------------------------
% \chapter{}

\end{apendicesenv}
% ---


% ----------------------------------------------------------
% Anexos
% ----------------------------------------------------------

% ---
% Inicia os anexos
% ---
\begin{anexosenv}

% Imprime uma página indicando o início dos anexos
\partanexos

% ----------------------------------------------------------
% Anexo A
% ----------------------------------------------------------
% \chapter{}


\end{anexosenv}

%---------------------------------------------------------------------
% INDICE REMISSIVO
%---------------------------------------------------------------------
\phantompart
\printindex
%---------------------------------------------------------------------

\end{document}
