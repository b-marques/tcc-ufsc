% The \phantomsection command is needed to create a link to a place in the document that is not a
% figure, equation, table, section, subsection, chapter, etc.
%
% When do I need to invoke \phantomsection?
% https://tex.stackexchange.com/questions/44088/when-do-i-need-to-invoke-phantomsection
\phantomsection
\chapter{Trabalhos Relacionados}
\phantomsection

Devido a importância dos esqueletos paralelos, e especifcamente o padrão paralelo estêncil, muitos esforços de pesquisas recentes buscam melhorar o desempenho e o suporte desses esqueletos em processadores \manycore. \textit{Buono} \etal\citeyear{buono13} portaram um \fw baseado em esqueletos paralelos, chamado \textit{FastFlow}, para o processador \textit{manycore} TilePro64, que possui 64 núcleos de processamento idênticos, interconectados por uma malha de \noc. Similarmente, \textit{Thorarensen} \etal\citeyear{thoraransen16} apresentaram um novo \textit{back-end} do \fw SkePU para o processador \textit{manycore} Myriad2, que possui como característica uma arquitetura heterogênea, visando dispositivos com restrição de energia e principalmente aplicações de visão computacional. \textit{Gysi} \etal\citeyear{gysi15} propuseram um \fw para otimização automática da repartição de computações estêncil em sistemas híbridos de \cpu e \gpu.

Recentes trabalhos estudaram o desempenho e/ou a eficiência energética de processadores \manycore de baixa potência. \textit{Totoni} \etal\citeyear{SCCEnergy:2012} compararam a potência e o desempenho do \textit{Intel's Single-Chip Cloud Computer} (SCC) com outros tipos de \textit{CPUs} e \textit{GPUs}. Porém, eles mostraram que não existe um solução única que entrega o melhor troca entre potência e performance, os resultados mostram que \textit{manycores} são uma oportunidade para o futuro. \textit{Souza} \etal\citeyear{Castro-Souza-CCPE:2016} propuseram um conjunto de \textit{benchmarks} para avaliar o \mppa \manycore processor. O \textit{benchmark} oferece diversas aplicações que utilizam padrões paralelos, tipos de trabalho, intensidade de comunicação e estratégias de carga de trabalho, adequado para uma ampla compreensão do desempenho e consumo de energia do \mppa e novos \textit{manycores} que estão por vir. \textit{Francesquini} \etal\citeyear{francesquini:hal-01092325} avaliaram três diferentes classes de aplicação (consumo de CPU, consumo de memória e uma composição híbrida dos dois tipos anteriores) utilizando plataformas de alto paralelismo como o \mppa em uma plataforma NUMA de 24 nós e 192 núcleos. Eles mostraram que as arquiteturas \textit{manycore} podem ser competitivas, mesmo se a aplicação é irregular por natureza.

De acordo com relevante conhecimento na área, o \pskelmppa é a primeira implementação completa de um \fw com uso de padrões paralelos no \mppa. A solução proposta \cite{Podesta:TCC} livra os programadores da necessidade de lidar explicitamente com a gestão de comunicação e envio de dados pela \noc, assim como a preocupação de lidar com um ambiente híbrido de execução e a ausência de coerência de \textit{cache} no \mppa.