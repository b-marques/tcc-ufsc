\swapcontents
% {
%     % Changing babel package inside a single chapter
%     % https://tex.stackexchange.com/questions/20987/changing-babel-package-inside-a-single-chapter
%     %
%     % Multiple-language document - babel - selectlanguage vs begin/end{otherlanguage}
%     % https://tex.stackexchange.com/questions/36526/multiple-language-document-babel-selectlanguage-vs-begin-endotherlanguage
%     \addtotextpreliminarycontent{English's Abstract}
%     \begin{otherlanguage*}{english}
%     \begin{resumo}[Abstract]

%         This is the english abstract.

%         \imprimirpalavraschave{Keywords}{\begin{inparaitem}[]\palavraschaveingles\end{inparaitem}}

%     \end{resumo}
%     \end{otherlanguage*}
% }
{
    \addtotextpreliminarycontent{Resumo em Português}
    \begin{otherlanguage*}{brazil}
    \begin{resumo}[Resumo]

        Uma nova classe de \textit{chips} altamente paralelos de baixo consumo energéitco que lidam com a restrição de energia foram descobertos. Os processadores Sunway SW26010 e Kalray MPPA-256 são exemplos deles, entregando mais de duzentos núcleos de processamento em um único \textit{chip} de baixo consumo energético. Apesar de apresentarem melhor eficiência energética do que os processadores \textit{multicore} de propósito geral, características arquiteturais como a limitada quantidade de memória ditribuída no \textit{chip} torna o desenvolvimento de aplicações científicas paralelas eficientes uma tarefa desafiadora. Neste projeto foram propostas otimizações ao \fw PSkelMPPA, que provê uma abstração única e de alto nível para programação estêncil no processador \mppa, livrando os programadores de serem responsáveis pela tarefa de explicitamente lidar com a comunicação e com o modelo de programação paralela híbrida do \mppa.

        \imprimirpalavraschave{Palavras-chaves}{\begin{inparaitem}[]\palavraschaveportugues\end{inparaitem}}

    \end{resumo}
    \end{otherlanguage*}
}



% % resumo em francês
% \addtotextpreliminarycontent{Français Résumé}
% \begin{resumo}[Résumé]
%   \begin{otherlanguage*}{french}
%       Il s'agit d'un résumé en français.

%       \imprimirpalavraschave{Mots-clés}{latex. abntex. publication de textes.}
%   \end{otherlanguage*}
% \end{resumo}


% % resumo em espanhol
% \addtotextpreliminarycontent{Español Resumen}
% \begin{resumo}[Resumen]
%   \begin{otherlanguage*}{spanish}
%       Este es el resumen en español.

%       \imprimirpalavraschave{Palabras clave}{latex. abntex. publicación de textos.}
%   \end{otherlanguage*}
% \end{resumo}



