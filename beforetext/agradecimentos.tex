\addtotextpreliminarycontent{\lang{Acknowledgement}{Agradecimentos}}

\begin{agradecimentos}

\lang
{
    Greetings.
}
{
    Aos meus pais por todo suporte,
    amor e dedicação, sempre priorizando uma educação de
    excelência para seus filhos. E que em mim depositaram
    toda sua confiança ao longo dessa jornada de 4 anos e meio.

    À minha namorada Samara Zimmermann, que me mostrou
    a força que um coração puro e generoso possui. Foi meu pilar
    emocional, que me incentivou e ajudou a manter o foco nos
    momentos mais difíceis. E me mostrou que desistir
    é algo que não existe no seu vocabulário.

    À Cecília, minha segunda mãe, um presente da vida,
    que está sempre a disposição e que eu amo incondicionalmente.

    Aos meus amigos, que compartilham comigo a aventura do mundo
    acadêmico da graduação, as dificuldades, as felicidades e toda
    a cumplicidade envolvida. Uma história que começou no ensino
    médio e foi estendida para a universidade e prosseguirá para a vida.
    
    Ao meu orientador Márcio Bastos Castro, seu profissionalismo
    e dedicação comprovou a existência de professores excepcionais
    e comprometidos com o objetivo de ensinar e produzir conhecimento
    científico. Sempre disposto e disponível, foi um exímio orientador.
    Você é minha referência profissional que levarei para a vida.
    
    E aos colegas de curso, que trilharam essa jornada ao
    meu lado e através da cumplicidade e companheirismo transpomos
    as barreiras que emergiam no decorrer do caminho.

    O presente trabalho foi realizado com o apoio do Conselho Nacional 
    de Desenvolvimento Científico e Tecnológico - CNPq - Brasil.
}

\end{agradecimentos}


%Mesmo padrão da seção primária, porém sem indicativo numérico. Assim como: Dedicatória, Resumo, Abstract, Sumário, Listas, Referências, Apêndices e Anexos.
%
%
%Corpo do texto, fonte 10,5, justificado, recuo especial da primeira linha de 1 cm, espaçamento simples.
%
