%% README.md
%% Copyright 2017 Evandro Coan
%
% This work may be distributed and/or modified under the
% conditions of the LaTeX Project Public License, either version 1.3
% of this license or (at your option) any later version.
% The latest version of this license is in
%   http://www.latex-project.org/lppl.txt
% and version 1.3 or later is part of all distributions of LaTeX
% version 2005/12/01 or later.
%
% This work has the LPPL maintenance status `maintained'.
%
% The Current Maintainer of this work is M. Y. Name.
%
% This work consists of the files:
% 1. `README.md`,
% 2. `basic.tex`,
% 3. `commands.tex`,
% 4. `commands_list.tex`
% 5. `programming.tex`
% 6. `badboxes.tex`



% Writing code in latex document. Usage: \begin & \end {lstlisting}
% http://stackoverflow.com/questions/3175105/writing-code-in-latex-document
\usepackage{listings}

% How to insert code with accents with listings?
% https://tex.stackexchange.com/questions/30512/how-to-insert-code-with-accents-with-listings
\usepackage{listingsutf8}

% set the font family for lstlisting
% https://tex.stackexchange.com/questions/33685/set-the-font-family-for-lstlisting
\usepackage{courier}

% Latex: Listings with monospace fonts
% https://stackoverflow.com/questions/2913141/latex-listings-with-monospace-fonts
\lstset{frame=,
  language=C++,% default language
  aboveskip=3mm,
  belowskip=3mm,
  showstringspaces=false,
  basicstyle={\small\ttfamily},
  numbers=left,
  numberstyle=\color{gray},
  keywordstyle=\color{blue},
  commentstyle=\color{darkgreen},
  stringstyle=\color{mauve},
  breaklines=true,
  breakatwhitespace=true,
  tabsize=4,
  morestring=[b]',
  morestring=[b]",
  literate = {---}{{\ProcessThreeDashes}}3
             {>}{{\textcolor{red}\textgreater}}1
             {|}{{\textcolor{red}\textbar}}1
             {\ -\ }{{\mdseries\ -\ }}3,
  inputencoding=utf8, % http://stackoverflow.com/questions/1116266/listings-in-latex-with-utf-8-or-at-least-german-umlauts
  extendedchars=true, % https://tex.stackexchange.com/questions/24528/having-problems-with-listings-and-utf-8-can-it-be-fixed
  literate=%
  {£}{{\pounds}}1
  {ß}{{\ss}}1
  {à}{{\`a}}1
  {À}{{\`A}}1
  {à}{{\`{a}}}1
  {á}{{\'a}}1
  {Á}{{\'A}}1
  {á}{{\'{a}}}1
  {Á}{{\'{A}}}1
  {â}{{\^a}}1
  {Â}{{\^A}}1
  {â}{{\^{a}}}1
  {Â}{{\^{A}}}1
  {ã}{{\~a}}1
  {Ã}{{\~A}}1
  {ã}{{\~{a}}}1
  {Ã}{{\~{A}}}1
  {ä}{{\"a}}1
  {Ä}{{\"A}}1
  {å}{{\r a}}1
  {Å}{{\r A}}1
  {æ}{{\ae}}1
  {Æ}{{\AE}}1
  {ç}{{\c c}}1
  {Ç}{{\c C}}1
  {ç}{{\c{c}}}1
  {Ç}{{\c{C}}}1
  {È}{{\'E}}1
  {è}{{\`e}}1
  {è}{{\`{e}}}1
  {é}{{\'e}}1
  {É}{{\'E}}1
  {é}{{\'{e}}}1
  {É}{{\'{E}}}1
  {ê}{{\^e}}1
  {Ê}{{\^E}}1
  {ê}{{\^{e}}}1
  {Ê}{{\^{E}}}1
  {ë}{{\"e}}1
  {Ë}{{\"E}}1
  {ë}{{\¨{e}}}1
  {ì}{{\`i}}1
  {Ì}{{\`I}}1
  {í}{{\'i}}1
  {Í}{{\'I}}1
  {í}{{\'{i}}}1
  {Í}{{\~{Í}}}1
  {î}{{\^i}}1
  {Î}{{\^I}}1
  {î}{{\^{i}}}1
  {Î}{{\^{I}}}1
  {ï}{{\"i}}1
  {Ï}{{\"I}}1
  {ò}{{\`o}}1
  {Ò}{{\`O}}1
  {ó}{{\'o}}1
  {Ó}{{\'O}}1
  {ó}{{\'{o}}}1
  {Ó}{{\'{O}}}1
  {ô}{{\^o}}1
  {Ô}{{\^O}}1
  {ô}{{\^{o}}}1
  {Ô}{{\^{O}}}1
  {õ}{{\~o}}1
  {Õ}{{\~O}}1
  {õ}{{\~{o}}}1
  {Õ}{{\~{O}}}1
  {ö}{{\"o}}1
  {Ö}{{\"O}}1
  {ø}{{\o}}1
  {ù}{{\`u}}1
  {Ù}{{\`U}}1
  {ù}{{\`{u}}}1
  {ú}{{\'u}}1
  {Ú}{{\'U}}1
  {ú}{{\'{u}}}1
  {û}{{\^u}}1
  {Û}{{\^U}}1
  {û}{{\^{u}}}1
  {ü}{{\"u}}1
  {Ü}{{\"U}}1
  {ő}{{\H{o}}}1
  {Ő}{{\H{O}}}1
  {œ}{{\oe}}1
  {Œ}{{\OE}}1
  {ű}{{\H{u}}}1
  {Ű}{{\H{U}}}1
  {€}{{\EUR}}1
}

% Defining `lstset` parameters for multiple languages & How can I highlight YAML code in a pretty way with listings?
%
% Usage \begin{lstlisting}[style=yaml_style] ... \end{lstlisting}
%
% https://tex.stackexchange.com/questions/45711/defining-lstset-parameters-for-multiple-languages
% https://tex.stackexchange.com/questions/152829/how-can-i-highlight-yaml-code-in-a-pretty-way-with-listings
\newcommand\YAMLcolonstyle{\color{red}}
\newcommand\YAMLkeystyle{\color{black}}
\newcommand\YAMLvaluestyle{\color{blue}}
\newcommand\ProcessThreeDashes{\llap{\color{cyan}\mdseries-{-}-}}

\lstdefinestyle{yaml_style}{
  frame=,
  aboveskip=3mm,
  belowskip=3mm,
  showstringspaces=false,
  numbers=left,
  numberstyle=\color{gray},
  breaklines=true,
  breakatwhitespace=true,
  tabsize=2,
  keywords={true,false,null,y,n},
  keywordstyle=\color{darkgray},
  basicstyle=\YAMLkeystyle,                                 % assuming a key comes first
  sensitive=false,
  comment=[l]{\#},
  morecomment=[s]{/*}{*/},
  commentstyle=\color{purple}\ttfamily,
  stringstyle=\YAMLvaluestyle\ttfamily,
  moredelim=[l][\color{orange}]{\&},
  moredelim=[l][\color{magenta}]{*},
  moredelim=**[il][\YAMLcolonstyle{:}\YAMLvaluestyle]{:}   % switch to value style at :
}

\lstdefinestyle{ufscthesisx_style}{
    aboveskip=3mm,
    belowskip=3mm,
    backgroundcolor=\color{white},   % choose the background color; you must add \usepackage{color} or \usepackage{xcolor}
    basicstyle={\small\ttfamily},    % the size of the fonts that are used for the code
    breakatwhitespace=true,          % sets if automatic breaks should only happen at whitespace
    breaklines=true,                 % sets automatic line breaking
    captionpos=t,                    % sets the caption-position to bottom
    commentstyle=\color{mygreen},    % comment style
    columns=flexible,
    deletekeywords={...},            % if you want to delete keywords from the given language
    escapeinside={\%*}{*)},          % if you want to add LaTeX within your code
    extendedchars=true,              % lets you use non-ASCII characters; for 8-bits encodings only, does not work with UTF-8
    frame=tb,                        % adds a frame around the code
    keepspaces=true,                 % keeps spaces in text, useful for keeping indentation of code (possibly needs columns=flexible)
    keywordstyle=\color{blue},       % keyword style
    language=Matlab,                 % the language of the code
    morekeywords={*,...},            % if you want to add more keywords to the set
    numbers=none,                    % where to put the line-numbers; possible values are (none, left, right)
    numbersep=5pt,                   % how far the line-numbers are from the code
    numberstyle=\tiny\color{mygray}, % the style that is used for the line-numbers
    rulecolor=\color{black},         % if not set, the frame-color may be changed on line-breaks within not-black text (e.g. comments (green here))
    showspaces=false,                % show spaces everywhere adding particular underscores; it overrides 'showstringspaces'
    showstringspaces=false,          % underline spaces within strings only
    showtabs=false,                  % show tabs within strings adding particular underscores
    stepnumber=2,                    % the step between two line-numbers. If it's 1, each line will be numbered
    stringstyle=\color{mymauve},     % string literal style
    tabsize=3,                       % sets default tabsize to 3 spaces
    texcl=true,                      % Permite o uso de acentuação no código
    title=\lstname                   % show the filename of files included with \lstinputlisting; also try caption instead of title
}
